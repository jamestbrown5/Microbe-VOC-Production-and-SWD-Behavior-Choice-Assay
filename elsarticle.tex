\documentclass[review]{elsarticle}

\usepackage{lineno,hyperref}
\modulolinenumbers[1]

\journal{A, B, C, or D}

%%%%%%%%%%%%%%%%%%%%%%%
%% Elsevier bibliography styles
%%%%%%%%%%%%%%%%%%%%%%%
%% To change the style, put a % in front of the second line of the current style and
%% remove the % from the second line of the style you would like to use.
%%%%%%%%%%%%%%%%%%%%%%%

%% Numbered
%\bibliographystyle{model1-num-names}

%% Numbered without titles
%\bibliographystyle{model1a-num-names}

%% Harvard
%\bibliographystyle{model2-names.bst}\biboptions{authoryear}

%% Vancouver numbered
%\usepackage{numcompress}\bibliographystyle{model3-num-names}

%% Vancouver name/year
%\usepackage{numcompress}\bibliographystyle{model4-names}\biboptions{authoryear}

%% APA style
%\bibliographystyle{model5-names}\biboptions{authoryear}

%% AMA style
%\usepackage{numcompress}\bibliographystyle{model6-num-names}

%% `Elsevier LaTeX' style
\bibliographystyle{elsarticle-num}
%%%%%%%%%%%%%%%%%%%%%%%

\begin{document}

\begin{frontmatter}

\title{Characterizing microbe volatile organic compounds responsible for \textit{Drosophila suzukii (Matsumura)} host choice}
\tnotetext[mytitlenote]{Fully documented templates are available in the elsarticle package on \href{http://www.ctan.org/tex-archive/macros/latex/contrib/elsarticle}{CTAN}.}

%% Group authors per affiliation:
%\author{James T. Brown\fnref{myfootnote}}
%\author{John J. Beck\fnref{myfootnote}}
%\address{1600/1700 SW 23rd Dr., Gainesville, FL. 32608}
%\fntext[myfootnote]{Since 1880.}

%% or include affiliations in footnotes:
\author[usda]{James T. Brown\corref{corresponding}}
\cortext[corresponding]{Corresponding author}
\ead{james.t.brown@usda.gov}

\author[uf]{Adam C.N. Wong}
\author[usda]{John J. Beck}

\address[usda]{USDA‐ARS Center for Medical, Agricultural, and Veterinary Entomology, 1700 SW 23rd Drive, Gainesville, FL 32608, USA}

\address[uf]{Department of Entomology and Nematology, University of Florida, Gainesville, FL 32611}



\begin{abstract}
The spotted-wing drosophila (\textit{Drosophila suzukii} (Matsumura)) has been a major insect pest in the continental United States since its arrival to California in 2008. These flies are capable of significant damage to cherries, blueberries, and other soft skin fruits. Currently, costs related to pest management and crop loss are estimated at \$718 million dollars annually. Management strategies for spotted-wing drosophila flies rely on heavily on broad spectrum chemical insecticides, but many of these measures fail due to reduced residual activity, low efficacy on spotted-wing drosophila, or dissipation by rainfall. Alternatively, host-fruit associated microbes have evolved to thrive in natural areas and the volatile metabolites produced by these microbes have been shown to influence host selection and oviposition preference in insect. Knowledge regarding the effect microbes and microbe volatile emissions have on host and host choice is limited; however, careful study of the chemical communication between spotted-wing drosophila and microbes maybe useful in predicting their impact on and spread to other crops. This information could also uncover possible microbial targets for biological control. Here, we characterize the volatile organic compounds produced by \textit{Frigirobactrium faeni} (K\"{a}mpfer et al. 2000) and \textit{Pantoe agglomerans} (Ewing and Fife 1972) microbe isolates to determine the  and response of spotted-wing drosophila to the microbes inoculated into blueberry juice.
\end{abstract}

\begin{keyword}
\textttInsect spotted-wing drosophia\sep VOC\sep blueberry juice\sep GC-MSD\sep 
\MSC[2019] 00-01\sep  99-00
\end{keyword}

\end{frontmatter}

\linenumbers
\begin{Introduction}
\section{Introduction}
\end{Introduction}

\begin{Materials and Methods}
\section{Materials and Methods}
\subsection{Spotted-Wing Drosophila Colony}
-Colony origins
-diet
Yield: 0.5L	Yield: 1.0	Yield: 1.5L
Water	0.3L	0.6L	0.9mL
Agar	4.33g	8.66g	13.0g
Brewer’s Yeast	12.5g	25.0g	37.5g
Corn meal	25.0g	50.0g	75.0g
Molasses	4.3g	8.66g	13.0g
Tegosept	0.5g	1.0g	1.5g
Propionic acid	2.66mL	5.33mL	8.0mL
Soy grits	4.33g	8.66g	13.0g


-rearing conditions
-
\subsection{Micorbe Culturing and Inoculation}
•	Setup traveling tray with:
o	Loops, culture tubes, pipettes, pipette bulb, marker, tube rack, media broth
o	Gloves, bleach, and EtOH, sample containers
•	Clean biohood with bleach then EtOH, wiping off the walls of the hood first and then the work surface. Next, clean everything with EtOH before placing it into the hood (including gloves)
•	Retrieve microbe:
o	Obtain desired bacteria for testing from glycerol tube.
	Remove glycerol stock from -80C freezer and transport to biohood frozen (dry ice or liquid N2)  make sure the glycerol tube is submerged up to the cap in liquid N2 to ensure that it remains at optimal temperature
o	Inoculate selected microbe into liquid media (LB/MRS/TSB, dependent upon which one it was isolated on originally). 
	Prepare culture tube with 6mL of broth media  first rinse the pipette with broth media before distribution 
	With a loop, scrape the glycerol stock tube and place loop into culture tube (repeat as needed).
	Return stock tube to freezing conditions AS QUICKLY AS POSSIBLE (the sample should not begin to thaw) 
	Before removing the loop from each culture tube, gently twist/shake in broth media to ensure that all the sample is off the loop
	Place the lids loosely on each of the culture tubes (past the first stop) & label each culture tube appropriately 
o	Incubate the sample in the shaking incubator for a 24-hour period at 32℃.
o	Prepare inoculation schedule and sampling schedule

Inoculation Day
•	After 24-hours, aliquot 200uL from each culture tube into 4 wells of a 96-well plate to obtain an absorbance reading.
•	Pellet bacteria and resuspend in culture broth solution
o	Centrifuge culture tubes at 6400 rcf for 10 minutes.
o	Remove supernatant
o	 Resuspend microbes in 10.2mL liquid broth into the culture tube
o	Use pipet and mix bacteria pellet of each sample. 
•	Add the total volume of the culture tube to the VOC sample container
•	Add 1.2mL of blueberry juice to VOC sampling container
•	Incubate sample container in a 32C incubator, and sample VOCs at 2 hours, 24 hours, and 48 hours after initial inoculation.

\subsection{VOC Collection and Analysis}
c.	SPME fiber selection
d.	Vent, P, and E time determination (vent= 10s, P=10m, E=5)
2.	Bacteria glycerol stocks on hand -80C freezer
a.	Bacillus Subtillus, Pantoe Agglomerans, Enterobacteria sp, Bacillus amloliquefacians, Frigirobactrium faeni
3.	VOC collected by HS-SPME, separated by GC-MS, at 2 hours, 24 hours, and 48 hours after microbe inoculation into fruit juice

\subsection{Choice Bioassay}
Objective to determine the degree to which EAD-active chemicals from wine and vinegar (without ethyl acetate) compare to the “luring” effect of these compounds in the field traps.
•	Traps
o	120mL container
o	Foil lid with hole for centrifuge tube
o	Drowning solution
o	Treatment dispensing container
•	Treatment dispensing Container
o	4ml poly vial
o	Cap with 3mm hole
o	Placed inside the cage 7mm apart 
o	
•	Arena
o	Metal cages with sleeves
	(20cm x 20cm x 20cm)
o	2 traps
o	Water soaked cotton ball
o	60-70 SWD
	Seven to 10 days old (even sex ratio)
	24h starved
•	Incubation conditions
o	21.5 C
o	23% rh
o	16:8 L:D
o	Flies checked at 1pm

\subsection{Statistical Analysis}
\end{Materials and Methods}

\begin{Results}
\section{Results}
\subsection{Insect Collection and Rearing}
\subsection{VOC Collection and Analysis}
\subsection{Choice Bioassay}
\subsection{Statistical Analysis}
\end{Results}

\begin{Discussion}
\begin{enumerate}
\end{enumerate}
\end{Discussion}

\bibliography{mybibfile}
\cite{Feynman1963118,Dirac1953888}

\end{document}